\documentclass[12pt]{article}
\usepackage{amsmath}
\usepackage{amssymb}
\usepackage{graphicx}
\usepackage{hyperref}
\usepackage[latin1]{inputenc}
\usepackage{listings}
\usepackage{xcolor}

\title{Documento di sintesi PENETRATION TESTING 2024}
\author{Simone Cappabianca - Mat: 5423306 \\  simone.cappabianca@edu.unifi.it}
\date{Febbraio 8, 2025}

\setlength{\parindent}{4em}
\setlength{\parskip}{1em}

\begin{document}
\maketitle
\newpage

\tableofcontents
\newpage


\section{Introduzione}

Questo documento riassume i concetti chiave, le metodologie e gli strumenti 
relativi al \textbf{penetration testing}, come delineato nei materiali del corso 
forniti. Il corso si concentra sulla fornitura di una \textbf{comprensione pratica 
dei principi dell'ethical hacking}, delle tecniche e della loro applicazione 
nella protezione delle infrastrutture IT. Il materiale pone l'accento 
sull'esperienza pratica all'interno di un ambiente di laboratorio controllato, 
incoraggiando un approccio pratico all'apprendimento della sicurezza. Il corso 
\`{e} tenuto dal professore associato Gabriele Costa presso IMT Lucca.

\section{Panoramica del Corso}
\begin{itemize}
    \item \textbf{Materiale Didattico}: Il corso utilizza \textbf{slides e 
    risorse open source} distribuite durante le lezioni. Gli studenti sono 
    incoraggiati a segnalare eventuali errori nel materiale fornito.
    \item \textbf{Configurazione del Laboratorio}: L'ambiente di laboratorio 
    \`{e} essenziale per il corso. Gli studenti creano un \textbf{ambiente 
    virtualizzato} usando strumenti come \text{VirtualBox, QEMU e Docker}. 
    Questo permette di eseguire simulazioni in modo sicuro.
    \item \textbf{Apprendimento Pratico}: Il corso enfatizza l'\textbf{esperienza 
    pratica} tramite esercizi e laboratori.
    \item \textbf{Coerenza Metodologica}: La metodologia per il penetration 
    testing rimane la stessa indipendentemente dal tipo di azienda che viene 
    valutata. 
\end{itemize}

\section{Temi Chiave}

\begin{enumerate}
    \item \textbf{Ethical Hacking e Penetration Testing}
    \begin{itemize}
        \item L'hacking \`{e} definito come \textit{"riutilizzare la tecnologia in modi 
        sorprendenti"}.
        \item L'hacking in ambito sicurezza \`{e} un'attivit\`{a} quotidiana 
        per ruoli difensivi e offensivi.
        \item L'enfasi \`{e} sull'utilizzo delle competenze di hacking per 
        \textbf{scopi etici}, come il penetration testing.
        \item Un \textbf{ethical hacker} non agisce mai di propria iniziativa, 
        ma in accordo con il cliente.
    \end{itemize}
    \item \textbf{Metodologia del Penetration Test}
    \begin{itemize}
        \item Il corso delinea una \textbf{metodologia standard} di penetration 
        testing che consiste nelle seguenti \textbf{7} fasi:
        \begin{enumerate}
            \item \textbf{Information gethering} (Raccolta di informazioni),
            \item \textbf{Network scanning} (Scansione della rete),
            \item \textbf{Enumeration} (Enumerazione),
            \item \textbf{Vulnerability assessment} (Valutazione delle 
            vulnerabilit\`{a}),
            \item \textbf{Exploitation} (Sfruttamento),
            \item \textbf{Post exploitation} (Post-sfruttament),
            \item \textbf{Final report} (Rapporto finale).
        \end{enumerate}
        \item Il materiale sottolinea che il processo pu\`{o} essere adattato a 
        seconda delle specifiche esigenze.
    \end{itemize}
    \item \textbf{Fondamenti di Networking}
    \begin{itemize}
        \item La comprensione dei concetti di rete \`{e} \textbf{fondamentale} 
        per l'ethical hacking e la sicurezza informatica.
        \item Il \textbf{modello ISO-OSI} \`{e} centrale per comprendere come 
        le reti operano. Il modello ISO-OSI \`{e} composto dai seguenti \textbf{7} 
        livelli:
        \begin{enumerate}
            \item \textbf{Application Layer}: protocolli di alto livello 
            interagiscono tra loro a questo livello. Alcuni di questi protocolli 
            includono (ma non sono limitati a) HTTP, HTTPS e FTP,
            \item \textbf{Presentation Layer},
            \item \textbf{Session Layer},
            \item \textbf{Transport Layer}: \`{e} qui che vengono effettivamente 
            gestiti gli scambi di dati. In questo livello discutiamo i due noti 
            protocolli TCP e UDP,
            \item \textbf{Network Layer}: questo livello si concentra sulla 
            comunicazione tra diverse reti. In questo caso, utilizzeremo 
            principalmente il protocollo Internet (IP),
            \item \textbf{Data Link Layer}: questo livello ha a che fare con le 
            reti locali (LAN), in una LAN \`{e} necessario fare riferimento 
            all'indirizzo MAC di un determinato dispositivo.,
            \item \textbf{Physical Layer}: questo livello include tutto ci\`{o} che 
            \`{e} correlato al trasferimento di dati all'interno di uno specifico 
            mezzo di comunicazione (rame, fibra ottica o onde radio)..
        \end{enumerate} 
        Il \textbf{Session layer} e \textbf{Presentation layer} non sono molto 
        rilevanti hai fine del corso.
        \item Segmentazione della reta:
        \begin{itemize}
            \item \textbf{LAN} (\textbf{Local Area Network}): rete locale (rete 
            aziendal),
            \item \textbf{WAM} (\textbf{Wide Area Network}): rete esterna 
            (internel),
            \item \textbf{DMZ} (\textbf{DeMilitarized Zone}): in quest'area 
            risiedono servizi che, pur essendo "interni", sono esposti al
            mondo esterno, proprio per i rischi impliciti, DMZ e LAN dovrebbero
            avere livelli di sicurezza diversi ed essere adeguatamente protetti.
        \end{itemize}
        \item Componenti chiave della rete:
        \begin{itemize}
            \item \textbf{Indirizzi MAC} (\textbf{Media Access Controll}): 
            identificano univocamente i dispositivi in una LAN.
            \item \textbf{Indirizzi IP}: identificano i dispositivi su una rete 
            e sono usati per la comunicazione su reti IP; possono essere privati 
            o pubblici.
            \item \textbf{ARP} (\textbf{Address Resolution Protocol}): si occupa 
            del collegamento tra l'indirizzo \textbf{MAC} e l'indirizzo \textbf{IP} 
            tramite la \textit{ARP Table}, questo \`{e} un protocollo 
            \textbf{cross-layer} (data link + network).
            \item \textbf{NAT} (\textbf{Network Address Translation}): traduce 
            gli indirizzi \textit{IP privati} in indirizzi \textit{IP pubblici} 
            per l'accesso a Internet.
            \item \textbf{DNS} (\textbf{Domain Name System}): traduce i nomi di 
            dominio in indirizzi IP.
            \item \textbf{DHCP} (\textbf{Dynamic Host Configuration Protocol}): 
            qusto protoollo assegna automaticamente tutti i parametri attraverso
            un server centrale che gestisce tutta la rete, un \textbf{DHCP} pu\`{o}
            generare due tipi di paccheti:
            \begin{itemize}
                \item \textbf{DHCPDISCOVER}: il PC che ha bisogno di un IP address
                spedisce questo pacchetto in broadcast sperando che il server DHCP 
                lo riceva,
                \item \textbf{DHCPOFFER}: quando il DHCP server riceve un pacchetto
                \textbf{DHCPDISCOVER} prova a soddisfare la richiesta spedendo
                un pacchetto \textbf{DHCPOFFER} con tutti i parametri necessari 
                per la configurazione.
            \end{itemize} 
            \item \textbf{Porte e Servizi}: usate per stabilire connessioni con 
            l'esterno e sono collegate ai servizi eseguiti su porte specifiche. 
        \end{itemize}
        \item Dispositivi di rete:
        \begin{itemize}
            \item \textbf{Switch}: Lo switch si basa sugli indirizzi MAC e 
            funziona all'interno di una specifica subnet. All'interno di una 
            subnet possiamo trovare uno o pi\`{u} switch.(ARP)
            \item \textbf{Router}: La funzione del router \`{e} quella di
            trasportare i pacchetti fuori dalla sottorete da cui provengono usando 
            gli indirizzi IP. (NAT)
            \item \textbf{Firewall}: Il firewall non \`{e} altro che un
            router con funzionalit\`{a} di sicurezza avanzate. Per convenzione,
            posizioniamo il firewall tra l'estremit\`{a} superiore del 
            \textit{Data link Layer} e l'estremit\`{a} inferiore del 
            \textit{Network layer}.
        \end{itemize}
        \item Strumenti di analisi di rete:
        \begin{itemize}
            \item \textbf{Wireshark}: analizza il traffico di rete catturando 
            pacchetti.
            \item \textbf{ARP, PING, Traceroute}: identificano problemi di 
            connettivit\`{a}, \textbf{ARP} lavora tra \textit{Data Link Layer} 
            (MAC Address) e \textit{Network Layer} (IP address) mentre 
            \textbf{PING} lavora sul \textit{Network Layer} (IP address).
        \end{itemize}
    \end{itemize}
    \item \textbf{Fasi del Penetration Testing in Dettaglio}
    \begin{enumerate}
        \item \textbf{Raccolta di Informazioni}:
        \begin{itemize}
            \item Raccogliere pi\`{u} informazioni possibili sull'obiettivo, 
            dal business agli strumenti utilizzati , dai fornitori, etc \dots
            \item Tecniche:
            \begin{itemize}
                \item \textbf{Google Dorking}: Utilizzo di operatori di ricerca 
                specifici per trovare informazioni sensibili.
                \item \textbf{Wayback Machine}: Per trovare dati storici.
                \item \textbf{Analisi dei social media}: Per individuare perdite 
                di informazioni (annunci di lavoro con le tecnomigia utilizzate).
                \item \textbf{Estrazione di metadati}: Un metadato non \`{e} 
                altro che informazioni aggiuntive inserite nel documento e pu\`{o}
                svolgere diversi scopi. (\textit{ExifTool}).
                \item \textbf{Query WHOIS}: Per informazioni sui domini. (\textit{WHOIS})
                \item \textbf{Query DNS}: Per la mappatura della rete.
                \item \textbf{Maltego e Recon-ng}: Strumenti per automatizzare 
                la raccolta dati.
                \item \textbf{Shodan}: \`{e} un potente motore di ricerca che ci 
                consente di trovare vulnerabilit\`{a} ed errori di configurazione 
                sui dispositivi esposti su Internet.
            \end{itemize}
            \item Strumenti utilizzati:
            \begin{itemize}
                \item ...
            \end{itemize}
        \end{itemize}
        
        \item \textbf{Scansione della Rete}:
        \begin{itemize}
            \item Identificare host attivi e porte aperte.
            \item Tipi di scansione:
            \begin{itemize}
                \item \textbf{ARP Scanning}: per scoprire dispositivi nella LAN.
                \item \textbf{ICMP/Ping Scanning}: per scoprire host e servizi attivi.
                \item \textbf{TCP Scanning}: include TCP Connect e SYN scans.
                \item \textbf{UDP Scanning}: identifica porte e servizi UDP aperti. 
            \end{itemize}
            \item Strumenti:
            \begin{itemize}
                \item \textbf{Nmap}.
            \end{itemize} 
        \end{itemize}
        \item \textbf{Banner Grabbing}:
        \begin{itemize}
            \item Determinare il servizio e la versione in esecuzione su una 
            porta specifica.
            \item Metodi: \textbf{Telnet, Netcat, Nmap}.
            \item Esempio HTTP: Utilizzo di comandi GET via Telnet.
            \item Nmap Service Probes utilizza espressioni regolari per 
            identificare servizi.
        \end{itemize}
        \item \textbf{Enumerazione}:
        \begin{itemize}
            \item Sfruttare le caratteristiche dei servizi per raccogliere 
            informazioni.
            \item Servizi comuni: 
            \begin{itemize}
                \item \textbf{SMTP},
                \item \textbf{DNS},
                \item \textbf{NETBIOS}.
            \end{itemize}
            \item Strumenti: 
            \begin{itemize}
                \item \textbf{script Nmap},
                \item \textbf{moduli Metasploit}.
            \end{itemize}
        \end{itemize}
        
        \item \textbf{Valutazione delle Vulnerabilit\`{a}}:
        \begin{itemize}
            \item Identificare e analizzare le debolezze della sicurezza.
            \item Metodi: 
            \begin{itemize}
                \item scanner automatici,
                \item database di vulnerabilit\`{a},
                \item  conoscenza del dominio.
            \end{itemize}
            \item Strumenti: 
            \begin{itemize}
                \item \textbf{Nessus},
                \item \textbf{Nexpose},
                \item \textbf{OpenVAS},
                \item \textbf{OWASP ZAP}.
            \end{itemize}
            \item Tipi di vulnerabilit\`{a}:
            \begin{itemize}
                \item \textbf{Cross-Site Scripting (XSS)}: sfrutta l'input 
                dell'utente per iniettare script malevoli.
                \item \textbf{SQL Injection (SQLi)}: sfrutta le vulnerabilit\`{a} 
                nelle query del database per accedere o modificare i dati.
            \end{itemize}
        \end{itemize}
        
        \item \textbf{Sfruttamento}:
        \begin{itemize}
            \item Strategie di attacco, movimenti laterali e shell remote.
            \item Strumenti: 
            \begin{itemize}
                \item \textbf{Netcat},
                \item \textbf{Metasploit},
                \item \textbf{BeEF}.
            \end{itemize}
            \item Tecniche: 
            \begin{itemize}
                \item Shell binding,
                \item reverse shell,
                \item sfruttamento lato client.
            \end{itemize}
        \end{itemize}
        \item \textbf{Post-Sfruttamento}:
        \begin{itemize}
            \item Ottenere persistenza, aumentare i privilegi e mappare la rete 
            interna.
            \item Metodi:
            \begin{itemize}
                \item creazione di servizi,
                \item modifica del registro di Windows,
                \item cracking degli hash delle password.
            \end{itemize} 
            \item Strumenti: 
            \begin{itemize}
                \item \textbf{moduli Metasploit},
                \item \textbf{John The Ripper}.
            \end{itemize}
        \end{itemize}
        \item \textbf{Rapporto Finale}:
        \begin{itemize}
            \item Presentare risultati, metodologie e piani di correzione.
            \item Deve includere:
            \begin{itemize}
                \item \textbf{sintesi},
                \item \textbf{metodologia},
                \item \textbf{risultati},
                \item \textbf{piano di correzione}.
            \end{itemize}
        \end{itemize}
    \end{enumerate}
    
\end{enumerate}




\section{Strumenti Chiave}
\begin{itemize}
    \item \textbf{Nmap}: scansione e enumerazione della rete.
    \item \textbf{Wireshark}: analisi del traffico di rete.
    \item \textbf{Netcat}: banner grabbing, shell binding e comunicazione di rete.
    \item \textbf{Virtualbox, QEMU, Docker}: creazione e gestione di ambienti virtuali.
    \item \textbf{Metasploit}: framework per exploitation e post-exploitation.
    \item \textbf{OpenVAS}: scansione automatica delle vulnerabilit\`{a}'.
    \item \textbf{John The Ripper}: cracking degli hash delle password. 
\end{itemize}


\section{Considerazioni sulla Valutazione delle Vulnerabilit\`{a}}
\begin{itemize}
    \item Le vulnerabilit\`{a} possono essere identificate automaticamente, 
    manualmente tramite database o tramite conoscenza del dominio.
    \item Gli scanner automatici possono tralasciare vulnerabilit\`{a} dipendenti 
    dall'applicazione, come XSS memorizzato.
    \item \textbf{Cross-Site Scripting (XSS)}: sfrutta l'input dell'utente per 
    iniettare script dannosi.
    \item \textbf{SQL Injection (SQLi)}: sfrutta le vulnerabilit\`{a} nelle query 
    del database per accedere o modificare i dati. Tecniche Blind SQLi possono 
    estrarre informazioni senza output diretto. 
\end{itemize}


\section{Conclusione}
Il materiale del corso fornisce una panoramica completa del penetration testing, 
dai concetti di rete alle tecniche di exploitation. Sottolinea l'esperienza 
pratica e un approccio strutturato all'apprendimento.

\end{document}