\documentclass[12pt]{article}
\usepackage{amsmath}
\usepackage{amssymb}
\usepackage{graphicx}
\usepackage{hyperref}
\usepackage[latin1]{inputenc}
\usepackage{listings}
\usepackage{xcolor}

\title{Documento di sintesi PENETRATION TESTING 2024}
\author{Simone Cappabianca - Mat: 5423306 \\  simone.cappabianca@edu.unifi.it}
\date{Febbraio 8, 2025}

\setlength{\parindent}{4em}
\setlength{\parskip}{1em}

\begin{document}
\maketitle
\newpage

\tableofcontents
\newpage


\section{Introduzione}

Questo documento riassume i concetti chiave, le metodologie e gli strumenti 
relativi al \textbf{penetration testing}, come delineato nei materiali del corso 
forniti. Il corso si concentra sulla fornitura di una \textbf{comprensione pratica 
dei principi dell'ethical hacking}, delle tecniche e della loro applicazione 
nella protezione delle infrastrutture IT. Il materiale pone l'accento 
sull'esperienza pratica all'interno di un ambiente di laboratorio controllato, 
incoraggiando un approccio pratico all'apprendimento della sicurezza. Il corso 
\`{e} tenuto dal professore associato Gabriele Costa presso IMT Lucca.

\section{Panoramica del Corso}
\begin{itemize}
    \item \textbf{Materiale Didattico}: Il corso utilizza \textbf{slides e 
    risorse open source} distribuite durante le lezioni. Gli studenti sono 
    incoraggiati a segnalare eventuali errori nel materiale fornito.
    \item \textbf{Configurazione del Laboratorio}: L'ambiente di laboratorio 
    \`{e} essenziale per il corso. Gli studenti creano un \textbf{ambiente 
    virtualizzato} usando strumenti come \text{VirtualBox, QEMU e Docker}. 
    Questo permette di eseguire simulazioni in modo sicuro.
    \item \textbf{Apprendimento Pratico}: Il corso enfatizza l'\textbf{esperienza 
    pratica} tramite esercizi e laboratori.
    \item \textbf{Coerenza Metodologica}: La metodologia per il penetration 
    testing rimane la stessa indipendentemente dal tipo di azienda che viene 
    valutata. 
\end{itemize}

\section{Temi Chiave}

\begin{enumerate}
    \item \textbf{Ethical Hacking e Penetration Testing}
    \begin{itemize}
        \item L'hacking \`{e}' definito come "riutilizzare la tecnologia in modi 
        sorprendenti".
        \item L'hacking in ambito sicurezza \`{e} un'attivit\`{a}' quotidiana 
        per ruoli difensivi e offensivi.
        \item L'enfasi \`{e} sull'utilizzo delle competenze di hacking per 
        \textbf{scopi etici}, come il penetration testing.
        \item Un \textbf{ethical hacker} non agisce mai di propria iniziativa, 
        ma in accordo con il cliente.
    \end{itemize}
    \item \textbf{Metodologia del Penetration Test}
    \begin{itemize}
        \item Il corso delinea una \textbf{metodologia standard} di penetration 
        testing che consiste nelle seguenti fasi:
        \begin{enumerate}
            \item Raccolta di informazioni.
            \item Scansione della rete.
            \item Enumerazione.
            \item Valutazione delle vulnerabilità.
            \item Sfruttamento.
            \item Post-sfruttamento.
            \item Rapporto finale.
        \end{enumerate}
        \item Il materiale sottolinea che il processo pu\`{o} essere adattato a 
        seconda delle specifiche esigenze.
    \end{itemize}
    \item \textbf{Fondamenti di Networking}
    \begin{itemize}
        \item La comprensione dei concetti di rete \`{e} \textbf{fondamentale} 
        per l'ethical hacking e la sicurezza informatica.
        \item Il \textbf{modello ISO-OSI} \`{e} centrale per comprendere come 
        le reti operano. (AGGIUNGERE il LIVELLI) 
        \item Componenti chiave della rete:
        \begin{itemize}
            \item \textbf{Indirizzi MAC}: identificano univocamente i dispositivi 
            in una LAN.
            \item \textbf{Indirizzi IP}: identificano i dispositivi su una rete 
            e sono usati per la comunicazione su reti IP; possono essere privati 
            o pubblici.
            \item \textbf{NAT}: traduce gli indirizzi IP privati in pubblici per 
            l'accesso a Internet.
            \item \textbf{DNS}: traduce i nomi di dominio in indirizzi IP.
            \item \textbf{DHCP}: assegna automaticamente gli indirizzi IP.
            \item \textbf{Porte e Servizi}: usate per stabilire connessioni con 
            l'esterno e sono collegate ai servizi eseguiti su porte specifiche. 
        \end{itemize}
        \item Dispositivi di rete:
        \begin{itemize}
            \item \textbf{Switch}: operano all'interno di subnet usando indirizzi 
            MAC.
            \item \textbf{Router}: inoltrano pacchetti tra reti diverse usando 
            indirizzi IP.
            \item \textbf{Firewall}: proteggono la rete applicando policy di 
            sicurezza.
        \end{itemize}
        \item Strumenti di analisi di rete:
        \begin{itemize}
            \item \textbf{Wireshark}: analizza il traffico di rete catturando 
            pacchetti.
            \item \textbf{ARP, PING, Traceroute}: identificano problemi di 
            connettivit\`{a}. 
        \end{itemize}
    \end{itemize}
    \item \textbf{Fasi del Penetration Testing in Dettaglio}
    \begin{enumerate}
        \item \textbf{Raccolta di Informazioni}:
        \begin{itemize}
            \item Raccogliere pi\`{u} informazioni possibili sull'obiettivo, 
            dal business agli strumenti utilizzati.
            \item Tecniche:
            \begin{itemize}
                \item \textbf{Google Dorking}: Utilizzo di operatori di ricerca 
                specifici per trovare informazioni sensibili.
                \item \textbf{Wayback Machine}: Per trovare dati storici.
                \item \textbf{Analisi dei social media}: Per individuare perdite 
                di informazioni.
                \item \textbf{Estrazione di metadati}: Dai file.
                \item \textbf{Query WHOIS}: Per informazioni sui domini.
                \item \textbf{Query DNS}: Per la mappatura della rete.
                \item \textbf{Maltego e Recon-ng}: Strumenti per automatizzare 
                la raccolta dati.
                \item \textbf{Shodan}: Per l'analisi delle vulnerabilit\`{a}.
            \end{itemize} 
        \end{itemize}
        
        \item \textbf{Scansione della Rete}:
        \begin{itemize}
            \item Identificare host attivi e porte aperte.
            \item Tipi di scansione:
            \begin{itemize}
                \item \textbf{ARP Scanning}: per scoprire dispositivi nella LAN.
                \item \textbf{ICMP/Ping Scanning}: per scoprire host e servizi attivi.
                \item \textbf{TCP Scanning}: include TCP Connect e SYN scans.
                \item \textbf{UDP Scanning}: identifica porte e servizi UDP aperti. 
            \end{itemize}
            \item Strumento: \textbf{Nmap}.
        \end{itemize}
        \item \textbf{Banner Grabbing}:
        \begin{itemize}
            \item Determinare il servizio e la versione in esecuzione su una 
            porta specifica.
            \item Metodi: \textbf{Telnet, Netcat, Nmap}.
            \item Esempio HTTP: Utilizzo di comandi GET via Telnet.
            \item Nmap Service Probes utilizza espressioni regolari per 
            identificare servizi.
        \end{itemize}
        \item \textbf{Enumerazione}:
        \begin{itemize}
            \item Sfruttare le caratteristiche dei servizi per raccogliere 
            informazioni.
            \item Servizi comuni: 
            \begin{itemize}
                \item \textbf{SMTP},
                \item \textbf{DNS},
                \item \textbf{NETBIOS}.
            \end{itemize}
            \item Strumenti: 
            \begin{itemize}
                \item \textbf{script Nmap},
                \item \textbf{moduli Metasploit}.
            \end{itemize}
        \end{itemize}
        
        \item \textbf{Valutazione delle Vulnerabilit\`{a}}:
        \begin{itemize}
            \item Identificare e analizzare le debolezze della sicurezza.
            \item Metodi: 
            \begin{itemize}
                \item scanner automatici,
                \item database di vulnerabilit\`{a},
                \item  conoscenza del dominio.
            \end{itemize}
            \item Strumenti: 
            \begin{itemize}
                \item \textbf{Nessus},
                \item \textbf{Nexpose},
                \item \textbf{OpenVAS},
                \item \textbf{OWASP ZAP}.
            \end{itemize}
            \item Tipi di vulnerabilit\`{a}':
            \begin{itemize}
                \item \textbf{Cross-Site Scripting (XSS)}: sfrutta l'input 
                dell'utente per iniettare script malevoli.
                \item \textbf{SQL Injection (SQLi)}: sfrutta le vulnerabilit\`{a} 
                nelle query del database per accedere o modificare i dati.
            \end{itemize}
        \end{itemize}
        
        \item \textbf{Sfruttamento}:
        \begin{itemize}
            \item Strategie di attacco, movimenti laterali e shell remote.
            \item Strumenti: 
            \begin{itemize}
                \item \textbf{Netcat},
                \item \textbf{Metasploit},
                \item \textbf{BeEF}.
            \end{itemize}
            \item Tecniche: 
            \begin{itemize}
                \item Shell binding,
                \item reverse shell,
                \item sfruttamento lato client.
            \end{itemize}
        \end{itemize}
        \item \textbf{Post-Sfruttamento}:
        \begin{itemize}
            \item Ottenere persistenza, aumentare i privilegi e mappare la rete 
            interna.
            \item Metodi:
            \begin{itemize}
                \item creazione di servizi,
                \item modifica del registro di Windows,
                \item cracking degli hash delle password.
            \end{itemize} 
            \item Strumenti: 
            \begin{itemize}
                \item \textbf{moduli Metasploit},
                \item \textbf{John The Ripper}.
            \end{itemize}
        \end{itemize}
        \item \textbf{Rapporto Finale}:
        \begin{itemize}
            \item Presentare risultati, metodologie e piani di correzione.
            \item Deve includere:
            \begin{itemize}
                \item \textbf{sintesi},
                \item \textbf{metodologia},
                \item \textbf{risultati},
                \item \textbf{piano di correzione}.
            \end{itemize}
        \end{itemize}
    \end{enumerate}
    
\end{enumerate}




\section{Strumenti Chiave}
\begin{itemize}
    \item \textbf{Nmap}: scansione e enumerazione della rete.
    \item \textbf{Wireshark}: analisi del traffico di rete.
    \item \textbf{Netcat}: banner grabbing, shell binding e comunicazione di rete.
    \item \textbf{Virtualbox, QEMU, Docker}: creazione e gestione di ambienti virtuali.
    \item \textbf{Metasploit}: framework per exploitation e post-exploitation.
    \item \textbf{OpenVAS}: scansione automatica delle vulnerabilit\`{a}'.
    \item \textbf{John The Ripper}: cracking degli hash delle password. 
\end{itemize}


\section{Considerazioni sulla Valutazione delle Vulnerabilit\`{a}}
\begin{itemize}
    \item Le vulnerabilit\`{a} possono essere identificate automaticamente, 
    manualmente tramite database o tramite conoscenza del dominio.
    \item Gli scanner automatici possono tralasciare vulnerabilit\`{a} dipendenti 
    dall'applicazione, come XSS memorizzato.
    \item \textbf{Cross-Site Scripting (XSS)}: sfrutta l'input dell'utente per 
    iniettare script dannosi.
    \item \textbf{SQL Injection (SQLi)}: sfrutta le vulnerabilit\`{a} nelle query 
    del database per accedere o modificare i dati. Tecniche Blind SQLi possono 
    estrarre informazioni senza output diretto. 
\end{itemize}


\section{Conclusione}
Il materiale del corso fornisce una panoramica completa del penetration testing, 
dai concetti di rete alle tecniche di exploitation. Sottolinea l'esperienza 
pratica e un approccio strutturato all'apprendimento.

\end{document}