\documentclass[12pt]{article}
\usepackage{amsmath}
\usepackage{amssymb}
\usepackage{graphicx}
\usepackage{hyperref}
\usepackage[latin1]{inputenc}
\usepackage{listings}
\usepackage{xcolor}

\title{ Report PENETRATION TESTING 2024}
\author{Simone Cappabianca - Mat: 5423306 \\  simone.cappabianca@edu.unifi.it}
\date{Febbraio 8, 2025}

\setlength{\parindent}{4em}
\setlength{\parskip}{1em}

\begin{document}
\maketitle
\newpage

\tableofcontents
\newpage

\section{Executive Summary}

Questo rapporto descrive i risultati di un penetration test condotto su un 
\textbf{container Docker}, identificato con l'indirizzo IP 172.19.0.3. Il test, 
effettuato in data 8 Febbraio 2025, aveva lo scopo di valutare la sicurezza del 
sistema e identificare potenziali vulnerabilit\`{a} sfruttabili. Le principali scoperte 
includono la presenza di una \textbf{vulnerabilit\`{a} di SQL Injection} nella procedura 
di login e una \textbf{vulnerabilit\`{a} di Cross-Site Scripting (XSS)} nella procedura 
di recupero password.

\section{Methodology Used}
\begin{itemize}
    \item \textbf{Definizione dello Scopo del Test:} Il test \`{e} stato eseguito su 
    un singolo container Docker, identificato dall'indirizzo IP 172.19.0.3. Non 
    \`{e} stata necessaria alcuna fase di raccolta di informazioni preliminare 
    poich\`{e} l'ambiente era preconfigurato.
    \item \textbf{Scansione della Rete:} La scansione di rete \`{e} stata eseguita 
    utilizzando \texttt{nmap} per identificare i servizi attivi. \`{E} stata condotta una 
    scansione ICMP/Ping di livello 3 e una scansione TCP SYN di livello 4.
    \item \textbf{Enumerazione:} Dopo la scansione, \`{e} stato effettuato il banner 
    grabbing tramite \texttt{nmap} e \texttt{telnet} per identificare il sistema 
    operativo e il web server. \`{E} stata eseguita una verifica per l'enumerazione 
    UserDir, risultata non attiva.
    \item \textbf{Valutazione delle Vulnerabilit\`{a}:} La valutazione delle 
    vulnerabilit\`{a} \`{e} stata condotta identificando manualmente la SQL Injection 
    nella procedura di login e la XSS nella procedura di password recovery.
    \item \textbf{Sfruttamento (Exploitation):} Sono state create delle Proof of 
    Concept (PoC) per le vulnerabilit\`{a} identificate, dimostrando la 
    possibilit\`{a} di sfruttarle.
    \item \textbf{Post-Sfruttamento:} Sono stati identificati script e query per 
    mantenere l'accesso e raccogliere dati, simulando le azioni di un attaccante 
    in una fase di post-exploitation.
    \item \textbf{Strumenti Utilizzati:} \texttt{nmap}, \texttt{telnet} e \texttt{Metasploit} 
    (per l'enumerazione UserDir, anche se il modulo non ha fornito risultati).
\end{itemize}

\section{Findings}
\begin{itemize}
    \item \textbf{Sistema Operativo:} Linux 4.15 - 5.8.
    \item \textbf{Web Server:} Apache/2.4.38 (Debian).
    \item \textbf{PHP:} PHP/7.2.34.
    \item \textbf{UserDir:} Il modulo \texttt{mod\_userdir} non \`{e} attivo sul 
    web server.
    \item \textbf{Vulnerabilit\`{a} di SQL Injection:}
    \begin{itemize}
        \item \textbf{Descrizione:} \`{E} presente una vulnerabilit\`{a} di SQL 
        Injection nella procedura di login che permette di manipolare la query 
        eseguita per il controllo delle credenziali. Inserendo un apice singolo 
        (') nel campo username, \`{e} possibile iniettare comandi SQL arbitrari.
        \item \textbf{Livello di Rischio:} Alto. La SQL Injection permette di 
        bypassare l'autenticazione e di recuperare informazioni dal database.
        \item \textbf{Proof of Concept (PoC):}
        \begin{enumerate}
            \item \textbf{Username:} \texttt{' OR null is null limit 1 \#}\\
            \textbf{Paswword:} Qualsiasi valore
            \item \textbf{Username:} \texttt{' UNION SELECT null, null, user() \#}\\
            \textbf{Password:} Qualsiasi valore
        \end{enumerate}
        Questi PoC dimostrano che \`{e} possibile manipolare la query SQL, ottenendo 
        informazioni sul database o bypassando l'autenticazione.
        \item \textbf{Evidenze:} Messaggio di errore SQL: \texttt{Notice: Invalid query: 
        You have an error in your SQL syntax; check the manual that corresponds 
        to your MariaDB server version for the right syntax to use near 
        'c9c35cf409344312146fa7546a94d1a6' at line 1 in /var/www/html/login.php 
        on line 63} viene mostrato quando viene inserito un apice singolo nel 
        campo username.
    \end{itemize}
    \item \textbf{Vulnerabilit\`{a}' di XSS:}
    \begin{itemize}
        \item \textbf{Descrizione:} \`{E} presente una vulnerabilit\`{a} di Cross-Site 
        Scripting (XSS) nella procedura di password recovery. Inserendo codice 
        JavaScript nel campo \textbf{Email address} \`{e} possibile eseguire codice 
        arbitrario nel browser dell'utente.
        \item \textbf{Livello di Rischio:} Medio. Permette l'esecuzione di codice 
        arbitrario nel browser dell'utente e potenzialmente il furto di cookie e 
        sessioni.
        \item \textbf{Proof of Concept (PoC):}
        \begin{enumerate}
            \item \textbf{Email address:} \texttt{<SCRIPT>alert('Hello')</SCRIPT>}
            \item \textbf{Email address:} \texttt{<IMG SRC="pic/1.jpg">}
        \end{enumerate}
        Questi PoC dimostrano la possibilit\`{a} di iniettare codice JavaScript nel 
        campo email.
        \item \textbf{Evidenze:} L'esecuzione di codice JavaScript tramite il 
        campo dell'email e i payload specificati dimostrano la vulnerabilit\`{a} XSS.
    \end{itemize}
\end{itemize}

\section{Remediation Plan}
\begin{itemize}
    \item \textbf{SQL Injection:}
    \begin{itemize}
        \item \textbf{Patch:} Utilizzare query parametrizzate (prepared statements) 
        per evitare che l'input dell'utente venga interpretato come codice SQL.
        \item \textbf{Mitigazione:} Implementare la validazione degli input per 
        impedire l'inserimento di caratteri speciali come l'apice singolo nelle 
        query SQL.
    \end{itemize}
    \item \textbf{XSS:}
    \begin{itemize}
        \item \textbf{Patch:} Implementare la sanificazione dell'input per rimuovere 
        o codificare i tag HTML che possono eseguire script.
        \item \textbf{Mitigazione:} Utilizzare Content Security Policy (CSP) per 
        limitare le fonti da cui il browser pu\`{o} caricare risorse.
    \end{itemize}
    \item \textbf{Misure aggiuntive:}
    \begin{itemize}
        \item Aggiornare regolarmente il sistema operativo, il server web e tutte 
        le applicazioni in uso per applicare le patch di sicurezza pi\`{u} recenti.
        \item Disabilitare o proteggere l'accesso a servizi non necessari.
        \item Implementare un firewall per limitare gli accessi non autorizzati.
    \end{itemize}
\end{itemize}

\section{Post-Exploitation (simulato)}

Sono stati identificati i seguenti script/query come esempio delle attivit\`{a} 
di un attaccante in una fase di post-exploitation:

\begin{itemize}
    \item Script per visuaizzare un file del server:\\
    \texttt{<SCRIPT>function test()\{fetch('TEST.txt').\\
    then(response => response.text()).then(data => alert(data))\}test(); \\
    </SCRIPT>}
    \item SQLi per recuperare l'elenco degli users/customers:\\
    \texttt{' or null is null INTO OUTFILE '/var/www/html/USERS.txt' \#}
    \item SQLi per recuperare l'elenco delle tabelle del db: \\
    \texttt{' or null is not null UNION SELECT null, TABLE\_NAME, TABLE\_SCHEMA \\ 
    FROM information\_schema.TABLES WHERE TABLE\_SCHEMA not like \\
    'information\_schema' AND TABLE\_SCHEMA not like 'mysql' AND TABLE\_SCHEMA \\
    not like 'performance\_schema' INTO OUTFILE '/var/www/html/TABLES.txt' \#}
\end{itemize}

Questi script dimostrano come un attaccante pu\`{o} accedere a file e informazioni 
dal database, utilizzando le vulnerabilit\`{a} SQLi e XSS.

\section{Conclusioni}

Questo rapporto evidenzia la presenza di vulnerabilit\`{a} critiche che richiedono 
immediata attenzione. Implementando le patch e le mitigazioni suggerite, la sicurezza 
del sistema pu\`{o} essere notevolmente migliorata.


\end{document}